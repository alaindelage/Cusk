% \documentclass[a4paper]{report}
\input{preamble}
\input{hyphenation}

%\setlength{\baselineskip}{2ex}
% \renewcommand{\baselinestretch}{1.2}
\selectlanguage{ngerman}
\usepackage{fourier}

\begin{document}
% \doublespacing
% \onehalfspacing
\thispagestyle{empty}

% einzeiliger Titel

% {\Large Rachel Cusk\ \ \emph{The Stuntman} (2023)}

% oder mehrzeiliger Titel

\begin{tabbing}
	{\Large Rachel Cusk} \ \ \={\Large \emph{The Stuntman}}\\[1ex]
	\> {\large (\emph{The New Yorker}, April 17, 2023)}
\end{tabbing}
\vskip 5ex

% \flushleft
\RaggedRight

\begin{quote}
	At a certain point in his career, the artist D, perhaps because he could find no other way to make sense of his time and place in history, began to paint upside down.\\[1ex]
	\textbf{This is how I imagine it.} 2
\end{quote}
\vskip 2ex

something disturbing about the female condition
\begin{quote}
	His wife believed that with this development he had \textbf{inadvertently expressed something disturbing about the female condition},\\[1ex] 
	and wondered if it might have repercussions in terms of his success, but the critical response to the upside-down paintings was enthusiastic 2
\end{quote}
\vskip 2ex

\begin{quote}
	The couple lived in a region of forests some distance from the city, for despite the world’s approval of him D had been angered and hurt by it and could not bring himself to forgive it. This is how I imagine it. 2
\end{quote}
\vskip 2ex

\begin{quote}
	It was because of the forests that D had found a way out of his artistic impasse, \textbf{caught as he had felt himself to be between the anecdotal nature of representation and the disengagement of abstraction}. 2
\end{quote}
\vskip 2ex

the notion of inversion
\begin{quote}
	He had spent a great deal of time observing the activities of the local foresters, and each time he saw a tree being felled this question of verticality had suggested itself to him. First, he had painted the men and the trees in a sort of joint condition of existence, in which the trunks were interchangeable with the bodies. Then he had seen how the bodies, too, could be felled, severed from their roots, and likewise turned on their side or cut into sections. The \textbf{notion of inversion} finally came to him as a means of resolving this violence and \textbf{restoring the principle of wholeness, so that the world was once more intact but upside down and thus free of the constraint of reality}. 2f.
\end{quote}
\vskip 2ex

acknowledge the existence of an unhappiness --- the possibility of madness as a kind of shelter
\begin{quote}
	When D’s wife first saw the upside-down paintings, she felt as though she had been hit. \textbf{This is how I imagine it}.\\[1ex] 
	\textbf{The feeling of everything seeming right yet being fundamentally wrong} was one she powerfully recognized. The paintings made her unhappy, or, rather, they led her to \textbf{acknowledge the existence of an unhappiness} that seemed always to have been inside her. D made a painting she particularly loved, of slender birch trees in sunlight, and the demented calmness and innocence of these upside-down trees seemed to \textbf{suggest the possibility of madness as a kind of shelter}. 3
\end{quote}
\vskip 2ex

this marginal perspective
\begin{quote}
	So he had come upon \textbf{this marginal perspective sidlingly, as it were, from a sideways direction, participating in its disenfranchisements, in its mute and broken identity, with the difference} that he had succeeded in \textbf{giving it a voice}. 3
\end{quote}
\vskip 2ex

\begin{quote}
	The early paintings were large portraits, fluid and somewhat naïve in style, of recognizable individuals from his region and from the circle of his acquaintance. [\dots\hspace{-0.3ex}]\\[1ex]
	Why were these people upside down? It was all one could ask, yet the answer seemed so obvious it felt as though any child could answer it, and so the paintings succeeded in illuminating a knowledge that the person looking at them already possessed.\\[1ex]
	D began to paint large, intricate landscapes in which nature seemed to be in its heyday [\dots\hspace{-0.3ex}]\\[1ex]
	It [nature] basked in a wordless moral plenitude, innocent and unconscious of the complete inversion it had undergone, and it was this quality of innocence, or ignorance, that succeeded in entirely \textbf{detaching the representational value of the painting from what it appeared to represent}. 3
\end{quote}
\vskip 2ex

\begin{quote}
	she noticed that he began to speak openly about his technique, explaining the difficulties of inverted painting that \textbf{could be resolved only through the use of photographs}. \textbf{Later, he rejected the photographic medium} and the paintings became even larger and more dreamlike and abstract.\\[1ex] 
	The question of what a human being actually was had never seemed so unanswerable. He often painted a man cowering alone in bed, the sullied oceanic blankness of the sheets, with the little tormented man somewhere at the top of the frame. 4
\end{quote}
\vskip 2ex

\begin{quote}
	D believed that women could not be artists. This is how I imagine it.\\[1ex] 
	As far as D’s wife knew, this was what most people believed, but it was unfortunate that he should be the one to say it out loud.\\[1ex] 
	She wondered whether it was her own indefatigable loyalty to him, her continual presence by his side, that had brought him to this view. Without her, he might still be an artist but he would not really be a man. He would lack a home and children, would lack the conditions for the obliviousness of creating, or, rather, would quickly be destroyed by that obliviousness.\\[1ex] 
	So she thought that what he was really saying was that women could not be artists if men were going to be artists. 4
\end{quote}
\vskip 2ex






% \changefont{ptm}{m}{n}













\end{document}